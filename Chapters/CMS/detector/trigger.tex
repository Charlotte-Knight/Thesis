\subsection{Trigger}\label{sec:trigger}

The beam crossing interval at the LHC is 25~\unit{ns}, corresponding to an event rate of 40~\unit{MHz}. With event sizes of about 1~\unit{MB}, this corresponds to a data rate of 38~\unit{TB/s}~\cite{CMS:2008xjf}. This is an unfeasible data rate to store and process, and therefore a trigger system is used to reduce the rate to a manageable level. The CMS trigger system operates in two stages: a level-1 (L1) trigger which brings the rate down to 100~\unit{kHz}, and a high-level trigger (HLT) which further reduces the rate to 1~\unit{kHz}. 

To provide a decision within the allowed latency of 3.2~\unit{\micro s}, the L1 trigger uses only coarse information about an event provided by the calorimeters and the muon system, and uses fast, custom-made electronics, primarily field-programmable gate arrays (FPGAs). A decision is made based upon the presence of high-\pt muons, electrons, tau leptons, photons, jets, and missing transverse energy. Information about the isolation of these objects can also be used. If an event passes the L1 trigger, the full event data is read out and passed to the HLT, which is software-based and runs on a CPU farm. The HLT is able to perform more complex reconstruction, and apply more sophisticated algorithms to make a trigger decision. 
There are hundreds of algorithms in the HLT (also referred to as HLT \textit{paths}) which are tuned to provide high signal efficiencies for the wide range of physics processes that the CMS experiment is interested in. The di-Higgs search described in \cref{chap:dihiggs} uses diphoton HLT paths which are described in more detail in \cref{sec:ggtt_trigger_preselection}.