\subsection{Coordinate System}
The CMS coordinate system has the origin centred at the nominal collision point, with the y-axis pointing vertically upwards, the x-axis pointing radially towards the centre of the LHC ring, and the z-axis pointing along the beam line in the direction of the counter-clockwise beam. An alternative, cylindrical coordinate system is also adopted, with the azimuthal angle $\phi$ measured from the x-axis in the x-y plane, the polar angle $\theta$ measured from the z-axis, and the radial coordinate $r$ measured from the beam line. Instead of $\theta$, pseudorapidity $\eta$ is often used, defined as:  
\begin{equation}
  \eta = -\ln[\tan(\theta/2)].
\end{equation}

Particles with high values of $\eta$ correspond to particles that are close to the beam line, and are said to be \textit{forward}. Distance between particles in $\eta$--$\phi$ space is measured by $\Delta R = \sqrt{\Delta\eta^2 + \Delta\phi^2}$.

Transverse momentum, \pt, is defined as the magnitude of the component of momentum perpendicular to the beam line, and is given by $\pt = \sqrt{p_x^2 + p_y^2}$. In many processes of interest, the outgoing particles are produced with a large \pt and therefore this is an important quantity.  Furthermore, since the incoming protons have zero \pt, the sum of \pt over all outgoing particles should be zero due to conservation of momentum. In cases where particles, such as neutrinos, are not detected, their total \pt can be inferred from the imbalance in the transverse momentum, which is called missing transverse momentum (\ptmiss). Similarly, transverse energy, \ET, is defined as $\ET=E\sin\theta$.