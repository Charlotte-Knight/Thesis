\subsection{Summary}
Object reconstruction at the CMS experiment is a complex process that involves dedicated algorithms for each type of object, often combining information from different subdetectors to maximize performance. All the reconstruction algorithms described in this Section have some relevance to the di-Higgs search presented in \cref{chap:dihiggs} but the most relevant are those for the photons and \tauh objects. The excellent ECAL energy resolution leads to a resolution of about 1\% on the diphoton invariant mass, \mgg, for values of \mgg around 125\GeV. In \Hgg analyses, this leads to a narrow signal peak in the \mgg distribution which is used to identify the signal above a large background. On the \tauh side, signal is separated from background using the Loose \Djet WP which identifies \tauh candidates with an 80\% efficiency whilst maintaining a jet misidentification rate between 1\% and 4\% depending on the \tauh \pt.