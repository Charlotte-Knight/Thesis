\subsection{Link Algorithm}\label{sec:link_algorithm}
After the creation of the PF elements (tracks and clusters), a linking algorithm is used to form particle candidates by creating associations between elements that are likely to have originated from the same particle~\cite{CMS:2017yfk}. Firstly, inner tracks are linked to calorimeter clusters when the track's extrapolated trajectory overlaps with the cluster. Similarly, clusters in different calorimeters are linked if the cluster's centre from the calorimeter with greater granularity is within the cluster of the calorimeter with coarser granularity. If there are multiple possible links, the link with the smallest distance in $\phi$-$\eta$ space, or $x$-$y$ space for ECAL-Preshower links, is chosen. Finally, links between tracks can be created if the tracks share a common secondary vertex.

Particle candidates are now formed sequentially, starting with muons, then electrons and photons, then charged and neutral hadrons~\cite{CMS:2017yfk}. Identification of muons and photons are also revisited at later stages to recover candidates that were missed by initially stringent criteria. At each stage, the PF elements related to the particle candidates are removed from the list of elements available to the link algorithm. This ensures that the algorithm does not link the same element to multiple candidates and that the same element is not used to create multiple candidates. Finally, an event post-processing step is performed to correct for artificially-large \ptmiss~\cite{CMS:2017yfk}.