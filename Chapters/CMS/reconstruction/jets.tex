\subsection{Jets}\label{sec:jet_reco}
Quarks and gluons (partons) produced in a proton-proton collision shower and hadronize into collimated sprays of particles. To make a measurement of an outgoing parton, these particles are first clustered into \textit{jets} and the parton's properties are inferred from properties of the jet, such as its energy. In most CMS analyses, including the di-Higgs search in \cref{chap:dihiggs}, jets are reconstructed using the \akt algorithm~\cite{Cacciari:2008gp,Cacciari:2011ma} with a distance parameter of 0.4. 

Jet energy corrections are derived from simulation studies so that the energy of reconstructed jets match the energy of particle-level jets, where the difference between reconstructed jets and particle-levels jets is whether the jet constituents are reconstructed or taken directly from the generator (truth-level). Further corrections are made using dijet, \gjet, and $\PZ+\text{jet}$ events to account for differences between data and simulation~\cite{CMS:2016ljj,CMS-DP-2021-033}. The jet energy resolution for jets with $\pt=30$\GeV is about 15--25\% depending on the level of pileup, and improves to 10--15\% at $\pt=100$\GeV and 5\% at $\pt=1\TeV$~\cite{CMS-DP-2021-033}.

The identification of jets originating from \Pqb quarks (\Pqb jets) is performed used a deep neural network algorithm, DeepJet~\cite{CMS-DP-2018-058}. The algorithm uses properties of the particle-flow constituents of a jet as well information from associated secondary vertices. Three working points are defined: loose, medium, and tight, corresponding to a rate of misidentifying a light jet as a \Pqb jet of about 10\%, 1\% and 0.1\% respectively. 