\subsection{Tracking}\label{sec:tracking}
The algorithm~\cite{Adam:2005cg,Cucciarelli:2006mt,CMS:2014pgm,Elmetenawee:2023xyl} for reconstructing tracks in the inner tracker is based on the combinatorial Kalman filter (CKF)~\cite{Fruhwirth:1987fm,Billoir:1989mh,Billoir:1990we,Mankel:1997dy}. Initially, tracking seeds are generated from a few hits consistent with a charged-particle trajectory. The seeds provide a coarse estimate of the track trajectory and from that, the CKF propagates through the tracker, looking for compatible hits in each tracker layer and updating the trajectory estimate as it does so. If multiple compatible hits are found in a layer, the CKF will create multiple track candidates and propagate them all (hence combinatorial). After exhausting all tracker layers, the tracks are refitted with greater precision and then selected based on quality criteria such as the number of hits and the $\chi^2$ of the fit. 

Given the combinatorial nature of the algorithm, CKF can be computationally expensive. This could be handled by, for example, only propagating tracks that have a high \pt or that coincide with the interaction point, but this would lead to a loss of efficiency for low \pt tracks or tracks that originate from displaced vertices. To mitigate this, an iterative approach is used where multiple runs of the CKF are performed, with varying types of seeds and selection criteria which target different types of tracks, for example, high \pt in one run and displaced tracks in the next. After each stage, the hits associated with the selected tracks are removed from the list of hits available to the CKF. This reduces the combinatorial complexity of the problem and allows the CKF to focus on the remaining hits. Ultimately, this iterative approach leads to higher reconstruction efficiency whilst keeping the misconstruction rate low~\cite{CMS:2017yfk}.

In the muon systems, a separate Kalman-filter based algorithm is used to reconstruct \textit{standalone-muon} tracks~\cite{CMS:2018rym}. These muon tracks can then be matched with inner tracks to create \textit{global-muon} tracks. Similarly, inner tracks are propagated to the muon system to create \textit{tracker-muon} tracks. Due to the high reconstruction efficiency in the inner tracker and muon systems, about 99\% of muons within the geometrical acceptance of the muon systems are reconstructed as global muons or tracker muons, and often as both~\cite{CMS:2017yfk}.

Electrons emit a sizeable fraction of their energy as bremsstrahlung photons before reaching the ECAL. If the momentum of an electron changes enough along its trajectory, the iterative CKF algorithm can fail to reconstruct the track. Therefore, for electrons, a Gaussian Sum Filter (GSF) algorithm is instead used which allows for more sudden and significant energy losses along the particle's path~\cite{Adam:2003kg}.