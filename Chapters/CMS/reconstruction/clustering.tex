\subsection{Calorimeter Clustering}\label{sec:calo_clustering}
Energy deposits in the calorimeters are grouped into \textit{clusters} where each cluster is hypothesized to originate from a single particle incident on the ECAL~\cite{CMS:2017yfk}. The clustering is performed independently in the ECAL barrel, ECAL endcaps, HCAL barrel, HCAL endcaps, Preshower and the HF. In the HF, the electromagnetic and hadronic components of each cell directly give rise to separate clusters. In the rest of the calorimeters, a more complex algorithm is used which is described below.

First, cluster seeds are identified as cells with energy larger than a given threshold and larger than any neighbouring cells, where neighbouring cells are the four cells that share a side with the seed cell or the eight cells that share a side or corner, depending on the calorimeter system. The seed cells are then grown into \textit{topological clusters} by adding cells that share a corner and have an energy of at least twice the noise level of the cell. 

Particles that are close to each other in $\eta$-$\phi$ space can have overlapping energy deposits, which in turn, can lead to a topological cluster having multiple seeds. In these cases, the energies from the cells are split and shared amongst the seeds. This is done using a Gaussian-mixture model that postulates that the energy deposits in the $M$ individual cells come from $N$ different Gaussian energy deposits, one for each seed. After fitting the model, the energy and positions of the $N$ Gaussian deposits are taken as cluster parameters~\cite{CMS:2017yfk}.

Electrons and photons have a significant probability of undergoing bremsstrahlung radiation or photon conversion prior to reaching the ECAL leading to several distinct clusters in the ECAL. To measure the energy of the original electron or photon, the clusters are grouped into a \textit{supercluster} which is then used to create the electron or photon candidate. A supercluster (SC) is first formed by including all clusters in an \ET-dependent zone in $\phi-\eta$ space centred around the seed cluster~\cite{CMS:2020uim}. Then, a conversion-finding algorithm~\cite{CMS:2008xjf} is used to identify tracks and associated ECAL deposits consistent with a photon conversion. Furthermore, at each tracker layer, tangents from an electron's trajectory are extrapolated to look for clusters from bremsstrahlung photons. All of this information is then fed into the PF link algorithm which builds electron and photon candidates.