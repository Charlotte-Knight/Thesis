\chapter{Conclusion}\label{chap:conclusion}
The Standard Model (SM) of particle physics is a remarkably successful theory, accurately describing the known fundamental particles and their interactions at the LHC and beyond. However, there are several key observations that the SM cannot explain, including dark matter~\cite{Clowe:2006eq}, neutrino oscillations~\cite{Super-Kamiokande:1998kpq}, and the matter-antimatter asymmetry of the universe~\cite{Canetti:2012zc}. Additionally, the SM has theoretical issues such as the hierarchy problem and the fact that it does not provide a description of gravity. Theories beyond the Standard Model (BSM) have been proposed that explain these anomalies and/or rectify the theoretical issues, but it is not clear which theory is correct. 

Such theories include Warped Extra Dimensions (WED), which alleviates the hierarchy problem, and the Next-to-Minimal Supersymmetric Standard Model (NMSSM), which solves the hierarchy problem and provides a candidate for dark matter. These theories were tested in \cref{chap:dihiggs} by searching for the resonant production of a pair of scalar bosons, a process that both theories predict. More specifically, WED predicts the production of a new boson, \PX, which can be spin-0 or spin-2 and decays to two SM Higgs bosons (\XHH), and the NMSSM predicts a new boson, \PX, which is only spin-0 and decays to a new scalar boson, \PY, and a SM Higgs boson (\XYH). 

The searches for \XHH and \XYH were performed using data collected by the CMS experiment between 2016 and 2018 at the LHC at $\sqrtS=13$\TeV corresponding to an integrated luminosity of 138\fbinv. Mass ranges of 260--1000\GeV and 50--800\GeV were considered for \mX and \mY respectively and for the first time at the CMS experiment, these searches were performed in final states including two photons and two tau leptons. This final state is particularly interesting given recent excesses reported in these decay channels for resonances around 95 and 650\GeV~\cite{CMS:2022bcb,CMS:2022goy,CMS:2024yhz}. In the \Ygg decay channel, a local excess corresponding to 2.3 standard deviations was observed at $(\mX,\mY) = (650,95)$\GeV which is not significant enough to claim a discovery, but does motivate the continued search for new resonances in this mass region, especially with new data collected at the LHC from 2022 and beyond. Other excesses were observed in the \XYttHgg, and low and high-mass \XYggHtt searches corresponding to local (global) significances of 2.6 (2.2), 3.2 (0.1), 3.2 (0.1) standard deviations respectively. Additionally, 95\% CL upper limits were placed on the cross sections of the \XHH and \XYH processes as functions of \mX and \mY, and theoretically excluded ranges of \mX and \mY were quoted for some of the searches.

Assuming that the WED or NMSSM theories are correct, it is possible that the new particles have masses larger than 1\TeV, and are therefore beyond the range of masses searched for. Similarly, the masses could be so large that the particles could not be produced at the LHC. In this case, a direct search for these particles at the LHC would be unfeasible. However, indirect effects of these particles could still be observed as small deviations in distributions of SM processes, which can be parameterized by the Wilson coefficients in an Effective Field Theory (EFT). 

In \cref{chap:eft}, these indirect effects were searched for with a SMEFT interpretation of a combination of Higgs boson measurements performed by the CMS experiment using data collected between 2016 and 2018 at $\sqrtS=13$\TeV corresponding to an integrated luminosity of 138\fbinv. The combination included an exhaustive variety of decay channels, and studied the production of the Higgs boson in a differential way using the STXS framework. This allowed for the individual constraint of 43 different Wilson coefficients where the tightest constraint corresponded to a probed energy scale up to 15\TeV when assuming $C_i=1$, and allowed for the simultaneous constraint of 17 linear combinations of Wilson coefficients where the tightest constraint corresponded to a probed energy scale up to 11\TeV. Furthermore, a discrepancy with respect to the SM was observed in the individual constraint on $C_{Hq}^{(3)}$ corresponding to a p-value of 0.01.

Two fundamentally different types of searches for new physics have been presented in this thesis. In both the direct and indirect searches, no evidence for new physics was observed at the crucial level of five standard deviations, but they have shown hints towards new physics, whether that be the 650\GeV and 95\GeV excess seen in the \XYH search, or the discrepancy in the $C_{Hq}^{(3)}$ Wilson coefficient. Looking forward, the CMS experiment will continue to take data, and will do so with unprecedented rates after the High-Luminosity LHC~\cite{ZurbanoFernandez:2020cco} becomes operational in the mid-2030s, resulting in a final integrated luminosity of 3000\fbinv at the end of the LHC's lifetime. This dataset represents a fantastic opportunity, and both direct and indirect searches should be utilized to maximize the chance of discovery new physics using it.