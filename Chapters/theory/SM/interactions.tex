\subsection{Interactions}
The electromagnetic and weak forces are described together by a single force: the \textit{electroweak} force. Particles that interact via the electroweak force include all fermions and the electroweak bosons: \PZ, \PW, $\gamma$ and \PH (all bosons except the gluon). Feynman diagram vertices for these interactions are given in \cref{fig:electroweak_fermion_vertices}. All fermions can interact via the exchange of \PZ boson and all left-handed particles/right-handed anti-particles can interact via the exchange of a \PW boson. All electrically charged fermions can interact via the exchange of a photon and all massive fermions can interact with the Higgs boson. The three and four-point interactions involving only electroweak bosons, shown in \cref{fig:electroweak_3_point_boson_vertices,fig:electroweak_4_point_boson_vertices}, are also allowed. In the strong force, quarks and gluons interact with each other according to the Feynman diagram vertices shown in \cref{fig:strong_vertices}.

\begin{figure}[b!]
  \centering
  \inputtikz{Figures/Theory/SM/ffz.tex}
  \inputtikz{Figures/Theory/SM/ffa.tex}
  \inputtikz{Figures/Theory/SM/ffh.tex} \\
  \inputtikz{Figures/Theory/SM/fnuw.tex}
  \inputtikz{Figures/Theory/SM/udw.tex}
  \caption[Electroweak Feynman Diagram Vertices]{Feynman diagram vertices allowed in the SM involving electroweak interactions with fermions. Interactions with a photon, \PZ boson, or a Higgs boson, require the fermion ($f$) flavour to be the same. Massive fermions are denoted by $f_m$ and exclude neutrinos. Interactions with \PW bosons can involve fermions from different generations and must involve either a lepton ($l$) and a neutrino ($\nu$) or an up-type quark ($u$) and a down-type quark ($d$). The $L$ and $R$ subscripts denote left-handed and right-handed fermions respectively. Charged-conjugated versions of the \PW-interaction diagrams are also possible if the handedness is reversed.}\label{fig:electroweak_fermion_vertices}
\end{figure} 


\begin{figure}
  \centering
  \inputtikz{Figures/Theory/SM/wwz.tex}
  \inputtikz{Figures/Theory/SM/wwh.tex}
  \inputtikz{Figures/Theory/SM/zzh.tex}
  \inputtikz{Figures/Theory/SM/hhh.tex}
  \caption[Three-Point Electroweak Boson Feynman Diagram Vertices]{Feynman diagram vertices allowed in the SM involving three electroweak bosons.}\label{fig:electroweak_3_point_boson_vertices}
\end{figure}

\begin{figure}
  \centering
  \inputtikz{Figures/Theory/SM/wwzz.tex}
  \inputtikz{Figures/Theory/SM/wwww.tex} \\
  \inputtikz{Figures/Theory/SM/wwhh.tex}
  \inputtikz{Figures/Theory/SM/hhhh.tex}
  \caption[Four-Point Electroweak Boson Feynman Diagram Vertices]{Feynman diagram vertices allowed in the SM involving four electroweak bosons.}\label{fig:electroweak_4_point_boson_vertices}
\end{figure}

\begin{figure}
  \centering
  \inputtikz{Figures/Theory/SM/qqg.tex}
  \inputtikz{Figures/Theory/SM/ggg.tex}
  \inputtikz{Figures/Theory/SM/gggg.tex}
  \caption[Strong Feynman Diagram Vertices]{Feynman diagram vertices allowed in the SM that involve a strong interaction. On the left, $q$ refers to a quark (or antiquark) of the same flavour.}\label{fig:strong_vertices}
\end{figure}