\subsection{Particle Content}
A schematic of the SM particle content, including every SM particle with its properties and categorizations is provided in \cref{fig:sm_particle_content}. The particle content can be initially categorized into two groups: spin-$\frac{1}{2}$ fermions which are matter constituents, and spin-integer bosons which are force carriers and are spin-1 except for the Higgs boson which is spin-0. 

There are four spin-1 bosons in the SM: the photon ($\gamma$) which mediates the electromagnetic force, the \PWpm and \PZ bosons which mediate the weak force, and the gluon (\Pg) which mediates the strong force. Particles that interact with photons carry \textit{electric charge}; particles that interact with gluons carry a \textit{colour charge}, which comes in three possible states: $r$, $g$ and $b$, and particles that interact with \PWpm and \PZ bosons carry charges called \textit{weak isospin} and/or \textit{weak hypercharge}. More details on these charges are given in \cref{sec:sm_lagrangian}.

The fermions are split by those that interact with the strong force, called \textit{quarks}, and those that do not, called \textit{leptons}. Leptons are further split into those which are electrically charged ($l$), and those which are not, called \textit{neutrinos} ($\nu$).  Quarks are similarly split by electric charge into up-type quarks and down-type quarks, with charge $\frac{2}{3}$ and $-\frac{1}{3}$ respectively (in units of elementary charge, $e$). 

The fermions can be also categorized into three generations based on a mass hierarchy, where the first generation is the least massive. Each generation contains an up-type quark, a down-type quark, a charged lepton, and a neutrino. The first generation contains the main constituents of the visible matter in the universe, those being the up and down quarks, and the electron. Additionally, all charged fermions have a corresponding antiparticle which has the same mass but opposite charge and parity.

The remaining particle is the Higgs boson which plays a special role in the SM. We will see later on that its introduction to the model is necessary to correctly describe the distribution of masses for particles in the theory. 

\begin{figure}
  \centering
  \hbox{\hspace{-2.5cm}\resizebox{1.2\textwidth}{!}{\inputtikz{Figures/Theory/SM/particle_content.tex}}}
  \caption[SM Particle Content]{Particle content of the SM. The mass, electric charge, colour and spin are given for each particle. All masses are taken from Ref.~\cite{ParticleDataGroup:2020ssz} except the Higgs boson mass which is taken from Ref.~\cite{CMS:2020xrn}.}\label{fig:sm_particle_content}
\end{figure}