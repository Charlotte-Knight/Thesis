\section{Sources of Systematic Uncertainty}\label{sec:ggtt_systematics}
In this analysis, systematic uncertainties affect the normalization and shape of the \mgg distribution for the single and di-Higgs processes. The other contributing processes: the nonresonant background, and in the case of the low-mass \XYggHtt analysis, the DY background, are fit entirely from data and have no associated systematic uncertainties. Normalization and shape changes due to systematic uncertainties are modelled in the likelihood with the techniques described in \cref{sec:stats_systematics}. The majority of systematic uncertainties affect only the normalization of the processes. This section details the sources of systematic uncertainty considered, and where appropriate, values that indicate the size of the uncertainty. Systematic uncertainties can be split into theoretical and experimental uncertainties, which are described in \cref{sec:ggtt_theoretical_uncert} and \cref{sec:ggtt_experimental_uncert} respectively. Further details about systematic uncertainties and their impact can be found in \cref{sec:ggtt_modelling} and \cref{sec:ggtt_results}.

\subsection{Theoretical Uncertainties}\label{sec:ggtt_theoretical_uncert}
Theoretical modelling of the single and di-Higgs processes have uncertainties related to QCD renormalization ($\mu_R$) and factorization ($\mu_F$) scales, the strong coupling constant ($\alpha_s$), and parton distribution functions (PDFs). These uncertainties enter the calculation of the inclusive cross sections for the single Higgs processes~\cite{LHCHiggsCrossSectionWorkingGroup:2016ypw}, which lead to an overall scaling of the normalization of single Higgs processes in every analysis category. The largest contribution comes from the uncertainty related to the $\mu_R$ and $\mu_F$ scales and is of the order of 5--10\%, depending on \PH production mode~\cite{LHCHiggsCrossSectionWorkingGroup:2016ypw}. These theoretical uncertainties also enter the MC simulation used to calculate the efficiency of each single and di-Higgs process in each analysis category, and lead to category-specific scalings of each process. However, the corresponding impacts on the efficiencies were found to be less than 1\% and were therefore not included in the final results extraction. There are also uncertainties on the \Hgg and \Htautau branching fractions whose values are around 2\%~\cite{LHCHiggsCrossSectionWorkingGroup:2016ypw}.

\subsection{Experimental Uncertainties}\label{sec:ggtt_experimental_uncert}

Experimental uncertainties consist mostly of uncertainties related to object calibration and the integrated luminosity measurement. The dominant sources of systematic uncertainty in this analysis are:
\begin{itemize}
    \item \textit{Photon energy calibration}: calibration of the photon energy scale and resolution that is described in \cref{sec:eg_energy_calibration} and in more detail in Refs.~\cite{CMS:2013lxn,CMS:2024ppo,CMS:2020uim}. The corresponding uncertainties affect both the rate and shape of the processes and the size of these impacts are described in \cref{sec:ggtt_modelling}.
    \item \textit{Preselection}: scale factors described in \cref{sec:ggtt_trigger_preselection} that are used to correct the efficiency of the preselection requirements in simulation to match data. The corresponding uncertainties on the signal efficiencies are between 4 and 5\%.
    \item \textit{Integrated luminosity}: integrated luminosities for the 2016, 2017 and 2018 data-taking periods, which have individual uncertainties between 1.2 and 2.5\%~\cite{CMS-PAS-LUM-17-004,CMS-PAS-LUM-18-002,CMS:2021xjt}.
    \item \textit{DeepTau ID scale factors}: scale factors described in \cref{sec:tau_deeptau_id} that are used to correct the efficiency of selections made using the DeepTau ID algorithm. The corresponding uncertainties on the signal efficiencies are up to 7\%.
    \item \textit{Photon ID scale factors}: scale factors described in \cref{sec:eg_id} that are used to correct the efficiency of selections made using the photon ID. The corresponding uncertainties on the signal efficiencies are up to 2\%.
    \item \textit{\tauh energy scale calibration}: calibration of the \tauh energy scale described in \cref{sec:tau_energy_scale} that leads to an uncertainty in the \tauh energy of between 0.6 and 0.8\%, depending on \tauh decay mode.
    \item \textit{Signal model interpolation}: interpolation of the signal efficiency at intermediate mass points. In each analysis category, an uncertainty is derived from the splines used in the interpolation that corresponds to a possible migration of signal events from one category to another. This is described further in \cref{sec:modelling_intermediate_mass_points}.
\end{itemize}
Additional systematic uncertainties are considered but have impacts on the final results less than ten times smaller than the sources described above. These include uncertainties related to trigger efficiencies, pileup simulation, electron identification, and the scale, resolution, and identification of muons, jets, and \MET.