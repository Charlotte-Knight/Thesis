\section{Summary}\label{sec:ggtt_conclusion}

This chapter presented the search for the resonant production of a pair of scalar bosons in the \ggtt final state via two primary processes: $\ppXHH$ where \PX is a new boson that can be spin-0 or spin-2, and $\ppXYH$ where \PX and \PY are new scalar bosons, and \PY is considered to decay to two photons or to two tau leptons. These two processes are motivated by the WED and NMSSM theories respectively. The searches were performed using data collected by the CMS experiment between 2016 and 2018 at the LHC at $\sqrtS=13$\TeV, corresponding to an integrated luminosity of 138\fbinv. 

Events were first selected with diphoton triggers and then a loose preselection on diphoton and ditau candidates was applied. Parametric neural network (pNN) classifiers were trained to separate signal events from background events in each of the \XHH and \XYH searches. The pNNs are parametric in \mX for the \XHH searches, and \mX and \mY for the \XYH searches, meaning that the pNN can specialize its classification behaviour to a chosen mass point, resulting in good performance at every mass point. The granularity in \mX and \mY that searches were performed at was optimized to reduce the possibility of missing an excess in the mass ranges for the search, which were 260--1000\GeV for \mX, and 50--800\GeV for \mY. At every mass point, analysis categories were defined based upon the pNN score, where the boundaries on the pNN score were optimized for the best expected sensitivity.

In every analysis category, statistical models were created to describe the \mgg distribution of the signal and background processes. The signal models were derived from simulated events, and interpolation methods were used to create models for mass points that were not simulated. The models of the SM single Higgs production background and of the nonresonant background were derived following typical techniques for \Hgg analyses at the CMS experiment, whereas the DY background originating from the misidentification of electrons as photons was modelled with a data-driven ABCD method developed specifically for this analysis. 

Observed significances and upper limits were extracted from maximum likelihood fits to the \mgg distributions in the analysis categories and these results are summarized in \cref{tab:ggtt_results_summary}. The largest excesses were found in the \XYggHtt searches at $(\mX, \mY) = (525, 155)$\GeV and $(\mX, \mY) = (450, 161)$\GeV corresponding to local significances of 3.2 standard deviations for both. The global significances for the \XYttHgg, and the low and high-mass \XYggHtt searches were 2.2, 0.1, and 0.3 standard deviations respectively. Therefore, as a standalone result, these searches do not provide any significant evidence for physics beyond the SM. However, given recent excesses reported in other CMS analyses, the excess at $(\mX,\mY) = (650,95)$\GeV in the \XYggHtt search is important.

\begin{table}
  \caption[Summary of Results from \XHH and \XYH Searches]{Summary of results from the \XHH and \XYH searches. The masses of the largest excesses and the corresponding local significances are shown as well as the global significance of the search where it has been calculated. The ranges of \mX that are excluded for the Radion, \XZero, and Graviton, \XTwo, are shown, as is the range of \mX in the low-mass \XYggHtt search where the observed limits are below their maximally-allowed values given experimental constraints~\cite{Ellwanger:2022jtd}.}\label{tab:ggtt_results_summary}
  \renewcommand{\arraystretch}{1.5}
  \begin{tabular}{@{}cccc@{}}
    \toprule
    Search                                                       & Largest Excess                                                                          & \begin{tabular}[c]{@{}c@{}}Global \\ Sig.\end{tabular} & \begin{tabular}[c]{@{}c@{}}Exclusion Range / \\ Limits Below \\ Maximally Allowed\end{tabular}                    \\ \midrule
    \XZeroHH                                                     & \begin{tabular}[c]{@{}c@{}}$1.3\sigma$ at\\ $\mX=366$\GeV\end{tabular}                  & ---                                                    & \begin{tabular}[c]{@{}c@{}}$\mX < 900$\GeV ($\Lambda_R=3$\TeV)\\ $\mX < 550$\GeV ($\Lambda_R=2$\TeV)\end{tabular} \\ \midrule
    \XTwoHH                                                      & \begin{tabular}[c]{@{}c@{}}$1.7\sigma$ at\\ $\mX=375$\GeV\end{tabular}                  & ---                                                    & $310 < \mX < 900$\GeV ($\tilde{\kappa}=1$)                                                                        \\ \midrule
    \XYttHgg                                                     & \begin{tabular}[c]{@{}c@{}}$2.6\sigma$ at\\ $(\mX, \mY) = (320, 60)$\GeV\end{tabular}   & 2.2\sigma                                              & ---                                                                                                               \\ \midrule
    \begin{tabular}[c]{@{}c@{}}Low-mass\\ \XYggHtt\end{tabular}  & \begin{tabular}[c]{@{}c@{}}$3.2\sigma$ at \\ $(\mX, \mY) = (525, 115)$\GeV\end{tabular} & 0.1\sigma                                              & \begin{tabular}[c]{@{}c@{}}$\mX < 620$\GeV \\ for $70 < \mY < 125$\end{tabular}                                   \\ \midrule
    \begin{tabular}[c]{@{}c@{}}High-mass\\ \XYggHtt\end{tabular} & \begin{tabular}[c]{@{}c@{}}$3.2\sigma$ at \\ $(\mX, \mY) = (450, 161)$\GeV\end{tabular} & 0.3\sigma                                              & ---                                                                                                               \\ \bottomrule
  \end{tabular}
\end{table}

In a similar \XYH search in the \bbgg final state, the CMS experiment reported an excess with a local significance of 3.8 standard deviations at $(\mX, \mY) = (650, 90)$\GeV~\cite{CMS:2023boe}. Furthermore, in searches for new scalars decaying to \WW, $\tau\tau$, and $\gamma\gamma$ final states the CMS experiment reported local significances of 3.8, 2.8 and 2.9 standard deviations for masses of 650\GeV, 100\GeV and 95\GeV respectively~\cite{CMS:2022bcb,CMS:2022goy,CMS:2024yhz}. For the search in the \ggtt final state described in this chapter, a similar excess was observed with a local significance of 2.3 standard deviations at $(\mX, \mY) = (650, 95)$\GeV. Put together, these results form a pattern that is worth investigating further, either from analyses in other final states, or from the analysis of future data collected at the LHC.

