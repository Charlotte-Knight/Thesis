\section{Summary}\label{sec:eft_conclusion}

Effective field theories (EFTs) are a powerful tool in the search for new physics. In the absence of the direct detection of new particles, EFTs provide a framework to parameterize the indirect effects that new physics can have on SM processes. In this Chapter, a SMEFT interpretation of a combination of Higgs boson analyses at the CMS experiment was presented. Most of the analyses in the combination were designed to measure the STXS, which studies the Higgs boson in a differential manner, splitting events according to the production mode, and by kinematic variables like \ptH. 

In \cref{sec:eft_parameterization}, the derivation of the parameterization was described, detailing the different choices required to achieve a valid parameterization for all production and decay modes. In the majority of cases, the scaling equations were derived including propagator corrections by using the \SMEFTsim UFO model to generate and reweight events to different values for the Wilson coefficients. For the \ggH and \ggZH production modes, the scaling equations were derived similarly, but with the \SMEFTatNLO model, allowing loop-level corrections to be included and ensuring that the parameterization is valid for the high \ptH STXS bins. For the \Hfl and \Hlnulnu decay channels, acceptance corrections were derived with a new standalone reweighting procedure, and dedicated samples were generated to derive their contributions to the total width of the Higgs boson. All possible Wilson coefficients under the \topUtl flavour assumption were considered, and the final scaling equations contained contributions from 43 different Wilson coefficients.

Results were reported on the individual constraints of 43 different Wilson coefficients or the simultaneous constraints of 17 linear combinations of Wilson coefficients. In the individual results, the tightest constraint corresponded to a lower limit on the probed energy scale of 15\TeV when assuming $C_i=1$, whereas in the simultaneous results, the tightest constraint corresponded to a lower limit of 11\TeV. Poor agreement with the SM was found in the individual constraint on $C_{Hq}^{(3)}$ corresponding to a p-value of 0.01 and a similar discrepancy was also found in the simultaneous fit for a linear combination of coefficients that heavily featured $C_{Hq}^{(3)}$. In both cases, the discrepancy was identified to be driven by excesses of events in high \ptV \WH and \ZH leptonic channels. 