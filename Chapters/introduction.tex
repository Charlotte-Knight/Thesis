\chapter{Introduction}\label{chap:intro}

It is the goal of particle physics to formulate a theory that accurately predicts the fundamental constituents of matter and their interactions. The Standard Model (SM) of particle physics~\cite{Glashow:1961tr,Weinberg:1967tq,Salam:1968rm} is the most successful theory to date, describing the electromagnetic, weak, and strong interactions. At its inception, the SM predicted the existence of a number of unseen fundamental particles, all of which have since been discovered. The last of these particles, the Higgs boson, was discovered in 2012 at the Large Hadron Collider (LHC) at CERN~\cite{CMS:2012qbp,ATLAS:2012yve} and marked the completion of the SM.  

Despite its many successes, the SM fails to describe several key observations including dark matter~\cite{Clowe:2006eq}, neutrino oscillations~\cite{Super-Kamiokande:1998kpq}, and the matter-antimatter asymmetry of the universe~\cite{Canetti:2012zc}. Therefore, the search for a theory beyond the Standard Model (BSM) that can also explain these observations is a key goal of modern particle physics. There are already a great number of theories that have been proposed, but it is not clear which theory is the correct one, so particle physics experiments continue to look for signatures of new physics that could point towards a particular theory or family of theories.

At the LHC, data continues to be taken in the search for new physics, now at higher collision energies and more intense luminosities than were needed for the Higgs boson discovery. Searches can be generally grouped into two categories: direct and indirect. Direct searches look for the on-shell production of a BSM particle and, if successful, would provide a measurement of the mass \textit{and} couplings of the new particle. Indirect searches, on the other hand, look for deviations from the SM predictions in the measurements of SM particles and processes, and would only provide information about the ratio of the new particle's couplings to its mass. Therefore, a direct observation of new physics is usually preferred, and often considered more convincing evidence. However, if the mass of the new particle is so large that it cannot be produced on-shell at the LHC, then indirect searches are the only option, making direct and indirect searches complementary to each other.

This thesis describes both a direct and an indirect search for new physics involving measurements of the Higgs boson at the LHC, specifically at the Compact Muon Solenoid (CMS) experiment, which is one of four major experiments situated around the LHC ring. The direct search is for a new boson, \PX, which decays to two SM Higgs bosons (\XHH), or to a new scalar boson, \PY, and to a SM Higgs boson (\XYH). In this thesis, this search will be referred to as the di-Higgs search. The indirect search uses measurements of Higgs boson cross sections, combined across Higgs boson decay channels, and interprets them in an Effective Field Theory (EFT) framework, which provides constraints on a variety of BSM scenarios. 

In \cref{chap:theory}, an introduction to the SM is provided, with a focus on the Higgs boson and its phenomenology. This chapter also introduces EFTs, and the BSM theories that motivate the direct search for \XHH and \XYH. In \cref{chap:cms}, the LHC and CMS experiment are described, and in \cref{chap:stats}, the statistical methods used in the searches are discussed. Finally, \cref{chap:dihiggs,chap:eft} describe the direct and indirect searches respectively, and a conclusion is drawn in \cref{chap:conclusion}.