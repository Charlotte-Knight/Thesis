\begin{tabular}{r|p{0.85\textwidth}}
    \toprule
    \multicolumn{2}{c}{\textbf{Isolation variables}} \\ \midrule
    $\mathcal{I}_{\rm{ph}}$ & Photon isolation, defined as the sum of transverse energy of PF photons falling inside a cone of radius ${\Delta}R=\sqrt{\Delta\eta^2+\Delta\phi^2}=0.3$ around the SC. \\
    $\mathcal{I}_{\rm{ch}}$ & Charged-hadron isolation, defined as the sum of transverse energy of the PF charged hadrons falling inside a cone of radius ${\Delta}R=0.3$ around the SC. \\
    $\mathcal{I}_{\rm{n}}$ & Neutral-hadron isolation, defined as the sum of transverse energy of all tracks in a hollow cone with a smaller (larger) annulus of ${\Delta}R=0.04$ (${\Delta}R=0.3$). \\ 
    $H/\Eraw$ & Ratio of the energy in the HCAL cells directly behind the SC to the energy of the SC. \\ \midrule
    \multicolumn{2}{c}{\textbf{Shower shape variables}} \\ \midrule
    $\RNINE$ & (=$E_{3\times3}/\Eraw$) The ratio of the energy sum in the $3\times3$ grid surrounding the SC seed to the energy of the SC before corrections. The value of \RNINE is typically high ($>0.85$) for unconverted photons, and typically lower ($<0.85$) for photons that have undergone a conversion upstream of the ECAL. \\
    $\sigma_{i\eta i\eta}$ & The standard deviation of the shower in $\eta$ in terms of the absolute number of crystal cells. \\
    $\sigma_{\eta}$ & A measure of the lateral extension of the shower, defined as the standard deviation of single crystal $\eta$ values within the SC, weighted by the logarithm of the crystal energy. \\
    $\sigma_{\phi}$ & A measure of the lateral extension of the shower, defined as the standard deviation of single crystal $\phi$ values within the SC, weighted by the logarithm of the crystal energy.  \\
    $\sigma_{RR}$ & For photons in the ECAL endcaps only. The standard deviation of the shower spread in the $x$-$y$ plane of the preshower detector. \\  \midrule
    \multicolumn{2}{c}{\textbf{Additional electron variables}} \\ \midrule
    $|1/E_{\text{SC}} - 1/p$| & The absolute difference between the inverse of the SC energy and the inverse of the momentum of the associated GSF track, where the inverse quantities are analogues for the track curvature. \\
    \bottomrule
\end{tabular}