\begin{titlepage}
  \begin{center}
      \vspace*{2cm}
      
      \huge{\textbf{Searching for direct and indirect signatures of new physics with measurements of the Higgs boson}}

      \vspace{1.5cm}
      \normalsize
      Charlotte Rose Knight
      
      \vspace{0.5cm}
      Imperial College London\\
      Department of Physics\\

      \vspace{5cm}
      A thesis submitted to Imperial College London\\
      for the degree of Doctor of Philosophy\\
      
  \end{center}
\end{titlepage}

\newgeometry{left=1.7in,right=1.7in,top=45mm,bottom=35mm}

\ftsection{Abstract}

This thesis describes two searches for new physics using measurements of the Higgs boson at the Compact Muon Solenoid (CMS) experiment. The first search looks for the direct production of a new boson, \PX, which decays to two Higgs bosons, or a Higgs boson and a new scalar boson, \PY. This search is performed in final states including two photons and two tau leptons, and considers \mX between 260 and 1000\GeV, and \mY between 50 and 800\GeV. The second search considers a scenario where new particles have masses too large to be produced on-shell at the LHC but where their existence can be probed instead by their indirect effects on measurements of single Higgs boson production. In this search, these effects are interpreted in the Standard Model Effective Field Theory (SMEFT).

The searches use data collected by the CMS experiment at the LHC between 2016 and 2018 in proton-proton collisions at a centre-of-mass energy of 13\TeV, corresponding to an integrated luminosity of 138\fbinv. In the direct search, an excess consistent with others seen by the CMS experiment is observed at $(\mX, \mY) = (650, 95)$\GeV with a local significance of 2.3 standard deviations, and other local excesses are observed up to 3.2 standard deviations, depending on \mX and \mY. Upper limits are also reported on the production cross sections, and ranges of \mX and \mY are excluded assuming different theoretical scenarios. In the indirect search, individual constraints on up to 43 SMEFT Wilson coefficients (WCs) are reported, where a discrepancy with respect to the SM prediction is observed for the $C_{Hq}^{(3)}$ WC, corresponding to a p-value of 0.01. Additionally, the simultaneous constraints on up to 17 linear combinations of WCs are reported.

\ftsection{Statement of Originality}

The work presented in this thesis is described in my words and references are cited where appropriate. The originality of the content is dependent on the chapter. \cref{chap:intro,chap:conclusion} provide an introduction and conclusion to the thesis and are entirely original.

\cref{chap:theory,chap:cms,chap:stats} provide background information that supports the work presented in the rest of the thesis. The content in these chapters is mostly not original, but is presented in my own words. 

\cref{chap:dihiggs} presents a resonant search for two scalar bosons performed by the CMS collaboration in the \ggtt final state. This work has been partly documented in Ref.~\cite{CMS:2025tqi} and a more complete description is provided in this thesis. I was predominantly responsible for this analysis, performing all work described in \cref{chap:dihiggs} except for the triggering and preselection described in \cref{sec:ggtt_trigger_preselection}.

\cref{chap:eft} presents a SMEFT interpretation of single Higgs boson production measurements made by the CMS experiment. This work has been partly documented in Ref.~\cite{CMS-PAS-HIG-21-018}. A more detailed description of the derivation of the EFT parameterization, which I was responsible for, is given in \cref{sec:eft_parameterization}. The measurements themselves, the combination of them, and the extraction of the final results, were performed by other members of the CMS collaboration. 

\ftsection{Copyright Declaration}

The copyright of this thesis rests with the author. Unless otherwise indicated, its contents are licensed under a Creative Commons Attribution-Non Commercial 4.0 International Licence (CC BY-NC). Under this licence, you may copy and redistribute the material in any medium or format. You may also create and distribute modified versions of the work. This is on the condition that: you credit the author and do not use it, or any derivative works, for a commercial purpose. When reusing or sharing this work, ensure you make the licence terms clear to others by naming the licence and linking to the licence text. Where a work has been adapted, you should indicate that the work has been changed and describe those changes. Please seek permission from the copyright holder for uses of this work that are not included in this licence or permitted under UK Copyright Law.

\ftsection{Acknowledgements}

I would like to thank the Imperial College London High Energy Physics group and the Science and Technology Facilities Council for giving me the opportunity to carry out this research. Thank you to my supervisor, Nick, for your support and invaluable insight into the world of particle physics. To Jon, thank you for your care and sustained and plentiful help throughout the PhD. To everyone else in the group and the wider CMS collaboration who has contributed along the way, thank you.

To my friends, thank you for the love and happiness that you brought and continue to bring into my life. In particular, I'd like to acknowledge Alie, Alina, Flax, and Victoria. During the more difficult times, our relationships kept me going, and I could not have completed this journey without you. To Hanae, I am so incredibly grateful for our friendship. You have been a constant source of support and joy, and meeting you has been one of the best things to come out of this PhD. To Mum, Dad and Laura, thank you for your life-long support and encouragement. I love you all.

\restoregeometry

\cleardoublepage
\pdfbookmark[0]{Contents}{Contents}
\tableofcontents

\cleardoublepage
\pdfbookmark[0]{List of Figures}{List of Figures}
\listoffigures

\cleardoublepage
\pdfbookmark[0]{List of Tables}{List of Tables}
\listoftables